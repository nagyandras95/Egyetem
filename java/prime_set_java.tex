\documentclass[12pt,a4paper]{article}
\usepackage[utf8]{inputenc}
\usepackage[magyar]{babel}
\usepackage[T1]{fontenc}
\usepackage{amsmath}
\usepackage{amsfonts}
\usepackage{amssymb}


\title{Prímhalmaz osztály megvalósítása Java osztály segítségével}
\begin{document}

Implementáljuk a természetes prímhalmaz (\textit{PrimeSet}) osztály Java nyelven. A prímhalmaz egy olyan halmaz, melyben csak prímek szerepelnek, és minden elem csak egyszer lehet benne. Egy ilyen halmaz reprezentálható egy természetes számmal, mely a halmaznak lévő elemek szorzatával egyezik meg. \\\\ 

Implementáljuk az alábbi műveleteket:
\begin{itemize}
\item \textit{contains}: eldönti egy természetes számról, hogy eleme-e a halmaznak. (osztható-e a reprezentáns elemmel)
\item \textit{insert}: beszúr egy prímet a halmazba (megszorozza a reprezentáns elemet az elemmel).
\item \textit{remove}: kivesz egy elemet a halmazból (leosztja az reprezentáns szorzatot elemmel)
\item \textit{write}:  írjuk ki a konzolra egy halmaz elemeit (nem kell optimálisnak lennie)
\item HF, plusz ha lesz rá idő: Számoljuk meg, hány elem van a halmazban.
\end{itemize} 

Oldjuk meg a segítségével az alábbi feladatot: Parancssori argumentumokban olvassunk be tetszőleges természetes számot, melyek akár ismétlődhetnek is. Írjuk ki ezek közül csak a prím számokat duplikátumok nélkül.

\end{document}