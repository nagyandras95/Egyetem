\documentclass[12pt,a4paper]{article}
\usepackage[utf8]{inputenc}
\usepackage[magyar]{babel}
\usepackage[T1]{fontenc}
\usepackage{amsmath}
\usepackage{amsfonts}
\usepackage{amssymb}
\begin{document}


\section{Feladat leírás}

Szimuláljunk egy paralementi választási rendszert. A szavazó polgárokról nyilván tartjuk a nevüket (szöveg), azonosítójuk (String) életkorukat (egész szám), nemeztétiségüket (felsorolási típus). \\ A szavazó polgárok három különböző pártok szavazhatnak (JavaHősök, CppFanok, HaskellMágusok). 

\\ A pártokra egy szavazórendszeren keresztül szavazhatnak, ahol nyilván tartjuk a pártokat, akikre szavazni lehet, illetve számoljuk a leadott szavazatokat. 
Szavazórendszer metódusai:
\begin{itemize}
\item szavaz: Paraméterként a szavazópolgárt, valamint egy pártot vár. Ha nem sikerült szavazás, dobjon a metódus kivételt.
\item osszSzavazat: Megadja, hogy összesen eddig hány polgár szavazott.
\item lezar: Lezárja a szavazást, mely után már nem lehet szavazni.
\item nyertes: Visszaadja, melyik párt a nyertes. Ha a szavazás még nincs lezárva, dobjunk kivételt.
\item hanySzavazat: Egy pártot vár paraméterként, és megadja, hogy egy adott pártra összesen eddig hányan szavaztak. Ha a szavazás még nincs lezárva, dobjunk kivételt.
\end{itemize}

A main hozzon létre három szavazópolgárt, és egy választási rendszert. Az első és a második polgár szavazzon a JavaHősök pátra, majd az első polgár próbáljon meg újra szavazni a CppFanokra. A második személy szavazzon a HaskellMágusokra. Zárjuk le a szavazást, írjuk ki, hogy összesen hányan szavaztak, és hogy ki a nyertes párt.

Hármasért: A szavazatokat egy \textit{HashMap} (kulcs-érték párokat tartalmazó típus) objektumban tároljuk, és ennek segítségével oldjuk meg, hogy ugyanazon azonosítóval ne szavazhasson egy polgár. Ilyenkor a szavaz metódus dobjon egy saját kivételt, melyben benne van, hogy melyik azonosítójú polgár próbált újra szavazni.

Négyesért: Egy polgár azonosítója lehessen egy tetszőleges típus, mely megvalósítja az \textit{Azonosito} interfészt. Az interfésznek legyen egy \textit{getKulcs} művelete, mely egy Stringet ad vissza. Ezt fogjuk eltárolni a \textit{HashMap}-ben. Hozzunk létre egy \textit{PolgarAzonosito} osztályt, mely megvalósítja az interfészt.

Ötösért: Az Azonosító interfész legyen generikus, melynek típusparamétere a \textit{getKulcs} visszatérési értéke legyen. Ennek megfelelően a \textit{Szavazórendszer} is legyen generikus osztály, melynek a \textit{HashMap} kulcsa a típusparaméter legyen. Csináljuk egy \textit{PolgarKulcs} osztályt, és érjük el, hogy tudjuk kulcsként használni (implementáljuk az equals s hashCode függvényeit).   

\end{document}