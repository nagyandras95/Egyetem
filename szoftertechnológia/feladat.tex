\documentclass[12pt,a4paper]{article}
\usepackage[utf8]{inputenc}
\usepackage[magyar]{babel}
\usepackage[T1]{fontenc}
\usepackage{amsmath}
\usepackage{amsfonts}
\usepackage{amssymb}

\title{Szoftvertechnológia feladatleírás}
\begin{document}
\maketitle
A félév során egy grafikus felületű, többszemélyes stratégiai játékot kell megvalósítani 3 fős csapatban, amely lehet körökre osztott (a játékosok felváltva cselekszenek), vagy valós idejű. Utóbbi esetben a játék hálózatos implementációja elengedhetetlen (a vezérlés miatt). Ugyanakkor a hálózatos implementációnál is megengedett a körökre osztott mód választása. \\

A hálózatos implementáció előfeltétele a jeles gyakorlati jegynek.

\section{Játék leírása}
Egy naprendszeren belül különböző méretű és hőmérsékletű bolygók vannak. A játékosok azonos feltételekkel indulnak. Van egy anyabolygójuk, melyen van szénbányájuk, naperőművük, mely a bánya energiaellását biztosítja, valamint rendelkeznek egy flottával, melyben egy darab anyahajó van. \\

Az nyer, akinek sikerül gyarmatosítania minden bolygót a naprendszerben. \\

A játékot egérrel lehessen kezelni. Bal gombbal kijelölhetjük a flottát, vagy bolygót. Jobb gombbal kijelölhetjük a kiválasztott flottánk célját. Ha ez egy hely, a flotta repüljön oda. Ha egy bolygó, akkor még ki kell választani a végrehajtandó akciót: landolás, szállítás, gyarmatosítás, ellenséges bolygó esetén támadás. Saját bolygót kijelölve a bolygót lehet fejleszteni, új épületet építeni rá, vagy meglévőt fejleszteni, ha van rajta elég hely. \\

\subsection{Játék alapelemei}
\begin{itemize}
\item \textbf{Erőforrások:} fém, deutérium (üzemanyag), 
\item \textbf{Bányák:} fémbánya, deutériumbánya
\item \textbf{Építmények:} Hajógyár, Naperőmű (bányák működtetéséhez)
\item \textbf{Hajók:} Anyahajó, Vadász, Csatahajó, Kolóniahajó, Szállítóhajó
\item \textbf{Akadályok:} Meteorit mező, Fekete lyuk
\item \textbf{Védelmi objektumok}: Rakétaállás, Lézerágyú, Ion Ágyú, Pajzskupola
\end{itemize}

\subsection{Hajók tulajdonságai}
\begin{tabular}{llllll}
\textbf{Képességek} & \textbf{Anyahajó} & \textbf{Vadász} & \textbf{Csatahajó} & \textbf{Kolóniahajó} & \textbf{Szállítóhajó} \\
\textbf{Fémköltség} & magas & alacsony & közepes & közepes & alacsony \\
\textbf{Fogyasztás} & magas & alacsony & közepes & magas & közepes \\
\textbf{Támadóerő} & közepes & alacsony & magas & - & - \\
\textbf{Életerő} & magas & alacsony & magas & közepes & alacsony \\
\textbf{Sebesség} & közepes & magas & közepes & alacsony & közepes

\end{tabular}


\subsection{Védelem tulajdonságai}
\begin{tabular}{lllll}
\textbf{Képességek} & \textbf{Rakétaállás} & \textbf{Lézerágyú} & \textbf{Ion Ágyú} & \textbf{Pajzskupola} \\
\textbf{Fémköltség} & alacsony & közepes & közepes & magas \\
\textbf{Támadóerő} & alacsony & közepes & magas & - \\
\textbf{Életerő} & alacsony & közepes & közepes & magas \\

\end{tabular}

\subsection{Játék alapeleminek jellemzői}

\textbf{Bolygók}: Épületeket/bányákat építhetünk rajta. Adott mérettel rendelkeznek, ami azt jelenti, hogy a rajta lévő épületeket csak egy bizonyos szintig lehet fejleszteni. A hőmérséklet azt befolyásolja, hogy egy naperőmű egy bizonyos szinten mennyi energiát termel.

\textbf{Fémbánya/Deutériumbánya}:  Ha van a bolygón elég energia, a játékos adott időközönként fémet/deutériumot kap. Ennek mennyisége a bánya szintjétől függ. \\

\textbf{Hajógyár}: Hajókat képes létrehozni a szintjétől függően.

\textbf{Naperőmű}: Szinjétől függően adott mennyiségű energiát szolgáltat a bolygó számára, melyet a bányák használhatnak fel. \\

\textbf{Anyahajó}: Ha a flotta rendelkezik anyahajóval, ellenséges bolygókat is képes megtámadni. \\

\textbf{Vadász, csatahajó}: Ha a flottában megtalálhatók, megtámadhatunk ellenséges flottákat. \\

\textbf{Szállító}: Fix kapacitással rendelkezik, erőforrásokat képes szállítani egyik bolygóról a másikra. \\

\textbf{Kolóniahajó}: Üres bolygókat képes gyarmatosítani, akár a flottában egyedül is állhat. Egy új bolygó gyarmatosításakor valamennyi fix erőforrást automatikusan lehelyez a bolygón.

\textbf{Védelmi egységek}: Egy bolygó megtámadása esetén beszállnak automatikusan a harcba, és segítenek a csatában. A pajzskupola nem képes támadni, viszont pozitívan befolyásolja a védelmi erejét a flottának.

Egy csata automatikusan zajlik le, a végeredmény a flotta hajóinak össz életerejétől és támadóerejétől függ. Ennek a megvalósítása többféleképpen történhet, például úgy, hogy körönként csapást mérnek egymásra az egyes hajók egy véletlenszerűen kiválasztott ellenséges hajóra.


\subsection{A játék skálázása}
\begin{itemize}
\item Akadályok és civil hajók (kolóniahajó, szállítóhajó) legalább három fős csapatok esetén kellenek csak. Bolygót elfoglalni bármilyen hajóval lehet, nyersanyagot megosztani pedig nem lehet bolygók közt.

\item Védelmi egységek csak 4 fős csapatok esetén kellenek. Négy fős csapatok esetén az épületnek van építési ideje: a bányáknak, naperőműveknek egy egység, a hajógyárnak két egység (az egység lehet kör, körökre osztott mód esetén, 1 másodperc valós idejű játék esetén). 
\end{itemize}

\end{document}