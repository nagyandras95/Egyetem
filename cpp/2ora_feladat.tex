\documentclass[12pt,a4paper]{article}
\usepackage[utf8]{inputenc}
\usepackage[magyar]{babel}
\usepackage[T1]{fontenc}
\usepackage{amsmath}
\usepackage{amsfonts}
\usepackage{amssymb}

\title{Feladat: C++ osztály készítése}
\begin{document}

\section{Racionális számokat ábrázoló osztály elkészítése}

Írjunk egy C++ osztályt, ami racionális számokat ábrázol. Egy racionális szám ábrázolható két egész szám hányadosaként, ezeket konstruktorparaméterekben adhatjuk meg. \\ \\
Írjuk meg az alábbi műveleteket:
\begin{itemize}
\item Készítsünk egy konstruktort, mely két egész számot vár. A hányados alapértelmezetten legyen egy.
\item \textit{add}, \textit{sub} művelet.
\item \textit{mul}, \textit{div} művelet.
\item \textit{eq} művelet, mely megmondja, hogy két racionális szám megegyezik-e.
\item A műveleteket lehessen láncba fűzni. pl: \textit{r.add(..).sub(..).add(..)};
\item Írjunk egy függvényt, mely egy racionális számokat tartalmazó \textit{vectort} vár, és összegzi az elemeit. 
\item Gyakorlás, HF: Készítsünk egy \textit{write} műveletet, mely egy \textit{iostream}-et vár, és beleírja a racionális szám értékét normál alakban (A két szám relatív prím egymáshoz képest, ehhez le kell osztani őket a legnagyobb közös osztóval).
\end{itemize}

\end{document}