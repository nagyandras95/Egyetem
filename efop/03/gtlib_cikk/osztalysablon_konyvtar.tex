\documentclass[11pt]{beamer}

\usepackage[utf8]{inputenc}
\usepackage[magyar]{babel}
\usepackage[T1]{fontenc}
\usepackage{lmodern}
\usepackage{zi4}
\usepackage{multirow}
\usepackage{listings}

\usetheme{Warsaw}

\renewcommand\UrlFont{\ttfamily\footnotesize}

% no navigation symbols
\setbeamertemplate{navigation symbols}{} 

% frame numbers
\expandafter\def\expandafter\insertshorttitle\expandafter{%
  \insertshorttitle\hfill%
  \insertframenumber\,/\,\inserttotalframenumber}

\author{Gregorics Tibor, Nagy András}
\title{Tapasztalatok az osztály-sablon könyvtár bevezetésével}
\date{2018. november 24.}

\logo{%
	\makebox[0.95\paperwidth]{%
		\hfill%
		\includegraphics[width=3.2cm,keepaspectratio]{efop-logo.jpg}%
	}%
}

\begin{document}

\begin{frame}
\titlepage
\end{frame}

\begin{frame}
	\frametitle{Osztálysablon könyvtár bemutatása}
	
	\begin{itemize}
		\item Nevezetes algoritmusokat leíró osztályokat tartalmaz
		\item Konkrét feladatok megoldása esetén leszármazással dolgozunk
		\item Az adott feladatra jellemző konkrét műveleteket és értékeket adhatjuk meg. (Pl. összegzésnél a neutrális elemet és az összegző műveletet)
		\item Valamint azt a gyűjteményt, amit feldolgozunk
	\end{itemize}
\end{frame}

\begin{frame}
	\frametitle{Az osztálysablon könyvtár bevezetése az oktatásba}
	\begin{itemize}
		\item Néhány éve vezettük be
		\vspace*{2px}
		\item A célja az újrafelhasználás gyakorlása volt
		\vspace*{2px}
		\item Vegyes fogadtatásban részesült, számos probléma felmerült a használata során, melynek nagy részét orvosoltuk
		\vspace*{2px}
		\item A pozitív tapasztalataink is voltak..
	\end{itemize}
\end{frame}

\begin{frame}
	\frametitle{Pozitívumok}
	\begin{itemize}

		\item Objektumorientált techinkák gyakorlása (leszármazás, függőség befecskendezés, stb..) 
		\vspace*{8px}
		\item Visszavezetéses techikák gyakorlása
		\vspace*{8px}
		\item C++ objektumorientált nyelvi elemek mélyítése, gyakorlása (sablonok, metódusok felül definiálása, stb..)
		
	\end{itemize}
\end{frame}

\begin{frame}
	\frametitle{Problémák, kritikák}
	\begin{itemize}
		\item Kritikák
		\begin{itemize}
			\item Nem ipari könyvtár
			\item Nagyobb kódméret
			\item Rossz konvenciók alapján íródott
		\end{itemize}
		\item Problémák
		\begin{itemize}
			\item Nem szakszerű használat
			\item Túl erős technikákra épül
			\item Egyre bővülő szabályrendszert kellett  bevezetni a használatára (mit lehet felüldefiniálni, kiből lehet származni..)
			\item Összegzés tétele túl általános volt
		\end{itemize}
	\end{itemize}
\end{frame}

\begin{frame}
	\frametitle{Összegzés átalakításai}
	\begin{block}{Új összegzés C++ kódja}

	\end{block}
\end{frame}

\begin{frame}
	\frametitle{Új felsorolók}
\end{frame}

\begin{frame}
	\frametitle{Új nyelvi elemek}
\end{frame}

\begin{frame}
	\frametitle{Összefoglalás, eredmények}
\end{frame}

\begin{frame}
	\frametitle{Tapasztalatok az osztály-sablon könyvtár bevezetésével}
	\begin{center}
		\Large{Köszönöm a figyelmet!}
	
	\vspace*{25px}
	{\small A projekt az Európai Unió támogatásával, az Európai Szociális Alap társfinanszírozásával valósult meg (EFOP-3.6.3-VEKOP-16-2017-00002).}\end{center}
\end{frame}

\end{document}
