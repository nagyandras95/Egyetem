\documentclass[11pt]{beamer}

\usepackage[utf8]{inputenc}
\usepackage[magyar]{babel}
\usepackage[T1]{fontenc}
\usepackage{lmodern}
\usepackage{zi4}
\usepackage{multirow}

\usetheme{Warsaw}

\renewcommand\UrlFont{\ttfamily\footnotesize}

% no navigation symbols
\setbeamertemplate{navigation symbols}{} 

% frame numbers
\expandafter\def\expandafter\insertshorttitle\expandafter{%
  \insertshorttitle\hfill%
  \insertframenumber\,/\,\inserttotalframenumber}

\author{Gregorics Tibor, Mózsi Krisztián, Szendrei Rudolf}
\title{Hogyan tanítsunk tesztelni?}
\date{2017. november 24.}

\logo{%
	\makebox[0.95\paperwidth]{%
		\hfill%
		\includegraphics[width=3.2cm,keepaspectratio]{efop-logo.jpg}%
	}%
}

\begin{document}

\begin{frame}
\titlepage
\end{frame}


\begin{frame}
	\frametitle{Miért fontos a tesztelés?}
	\begin{itemize}
		\item a minőségbiztosítás egyik legfontosabb eszköze
		\vspace*{2px}
		\item a programozók emberek, hibáznak % kódolási hibák / a hibák felderítésének módja pedig a tesztelés
		\vspace*{2px}
		\item ipari környezetben tipikusan automatikusan futtatható fejlesztői tesztek készülnek (egység és integrációs tesztek)
		\vspace*{2px}
		\item ...
	\end{itemize}
\end{frame}

\begin{frame}
	\frametitle{A cél}
	\begin{itemize}

		\item megmutatni a hallgatóknak a tanulmányaik korai szakaszában a tesztelés \textbf{szükségességét}, előnyeit 
		\vspace*{8px}
		\item a kezdő programozók tanításának egy alternatívája az \textit{analóg programozás}, ennek egyik változata a visszavezetés
			\begin{itemize}
				\item egy-egy programozási tételhez összegyűjteni a javasolt megvizsgálandó teszteseteket 
				\item így \textbf{szabványt} adhatunk a programozó kezébe 
				\item ezzel a tesztlefedettséget növelhetjük 
			\end{itemize} 
		\vspace*{8px}
		\item \textbf{konkrét eszköz} ajánlása az \textit{automatizált} teszteléshez
		
	\end{itemize}
\end{frame}

\begin{frame}
	\frametitle{Esettanulmány}
	\begin{block}{Feladat}
		Válasszuk ki egy egész számokat tartalmazó mátrix azon sorát, melynek a legnagyobb a sorösszege.
	\end{block}
	\vspace*{6px}
	\textbf{Megoldó program előállítása}
	\begin{itemize}
		\item analóg programozás, visszavezetés technikáját használva
		\item ez a választás hatással van a programkészítés teljes életciklusára
		\item egyszerű programok készítésének fázisai: elemzés, tervezés, megvalósítás, \textit{tesztelés}
	\end{itemize}
	\vspace*{10px}
\end{frame}

\begin{frame}
	\frametitle{Elemzés}
	\begin{block}{Specifikáció}
		\textbf{Változók}:\hspace{13px}(data:$\mathbb{Z}^{nxm}$, maxSumIndex:$\mathbb{N}$, maxSum:$\mathbb{Z}$)
		\smallskip\linebreak
		\textbf{Előfeltétel}:\hspace{7px}(data = data' $\wedge$ n > 0)
		\smallskip\linebreak
		\textbf{Utófeltétel}: (Előfeltétel $\wedge$ maxSum = $MAX_{i=1}^n$ rowSum(i) $\wedge$
		\linebreak
		\hspace*{122px} maxSum = rowSum(maxSumIndex))
		\bigskip\linebreak
		ahol $rowSum: \mathbb{N} \rightarrow \mathbb{Z}$\linebreak	
		\hspace*{19px} $rowSum(i) = \sum_{j=1}^m data[i,j]$
	\end{block}
	
	\begin{itemize}
		\item már az elemzés során az alkalmazandó programozási tételekről gondolkodhatunk (maximum kiválasztás, összegzés)
	\end{itemize}
	\vspace*{20px}
\end{frame}

\begin{frame}
	\frametitle{Tervezés}
	\textbf{Tervezés}
	\begin{itemize}
		\item a kiválasztott két programozási tételt kell adaptálni a konkrét részfeladatokra
		\item az így kapott algoritmusokat kell megfelelően összeilleszteni
	\end{itemize}
\end{frame}


\begin{frame}
	\frametitle{Megvalósítás}
	\textbf{Megvalósítás}
	\begin{itemize}
		\item procedurális megoldást adunk
	\end{itemize}
	\begin{block}{A megvalósított alprogramok szignatúrái}
		\texttt{void maxRowSum(const vector<vector<int>> \&data, \\
			\hspace*{82px}int \&maxSumIndex,\\
			\hspace*{82px}int \&maxSum); \medskip\\
			int rowSum(const vector<int> \&row);}
	\end{block}
\end{frame}


\begin{frame}
	\frametitle{Tesztelés}
	\begin{itemize}
		\item a végrehajtható specifikáció alapján készülő tesztesetek között vannak olyanok, amik már belelátnak a megoldó programba is (\textbf{szürke doboz esetek}) 
		\vspace*{8px}
		\item szabványosítható a tesztelés
		\vspace*{8px}
		\item kiegészítve néhány ismert fekete doboz tesztesettel (érvénytelen bemenet) már jó lefedettséget ad 
	\end{itemize}
\end{frame}

\begin{frame}
	\frametitle{Szürke doboz tesztesetek}
	\textbf{Feldolgozott gyűjteményre vonatkozó esetek}
	\begin{itemize}
		\item a programozási tételek közösek abban, hogy egy gyűjtemény felsorolására épülnek
		% ami sokminden lehet, pl halmaz, intervallum, sorozat, vagy akár seq input fájl
	\end{itemize} \vspace*{-5px}
	\begin{center}
		\begin{tabular}{ | l | l | r | }
			\hline
			\multirow{2}{100px}
			{Gyűjtemény határainak vizsgálata} & Első elem feldolgozásra kerül-e \\
					& Utolsó elem feldolgozásra kerül-e \\
					%?& Közbenső elem feldolgozásra kerül-e \\
			\hline
			\multirow{3}{*}
			{Gyűjtemény mérete szerint} & \textit{Üres gyűjtemény kezelése} \\
					& Egy elemű gyűjtemény kezelése \\
					& Több elemű gyűjtemény kezelése \\
			\hline
		\end{tabular}
	\end{center}
	
	
	Maximum kiválasztás:
	\begin{itemize}
		\item a határok vizsgálatához két tesztadat szükséges %az egyiknél az első, a másiknál az utolsó elem a maximális%
	\end{itemize}
	Összegzés:
	\begin{itemize}
		\item a határok vizsgálatához elegendő egy tesztadat %azt ellenőrizzük, hogy vajon minden elemet figyelembe vettünk
	\end{itemize}
\end{frame}


\begin{frame}
	\frametitle{Szürke doboz tesztesetek}
	\textbf{Programozási tételek speciális tesztesetei}
		
	\begin{center}
		\begin{tabular}{ | l | l | r | }
			\hline
			\multirow{2}{100px}
			{Maximum kiválasztás} & Felsorolás közepén lévő maximális elem \hspace*{-1px}\\
			& Több azonos maximális elem \\
			\hline
			Összegzés & Terheléses teszt \\
			\hline
		\end{tabular}
	\end{center}
	
		\begin{center}
			\begin{tabular}{ | l | l | r | }
				\hline
				\multirow{2}{100px}{Számlálás}
				& Az adott tulajdonságnak nulla, egy, \\
				& kettő vagy több elem tesz eleget\\
				\hline
				\multirow{2}{100px}{Keresés}
				& Létezik/nem létezik a keresett tulajdon- \\
				& ságnak megfelelő (közbenső) elem\\
				\hline
			\end{tabular}
		\end{center}
		...
	
	\begin{itemize}
		\item ezen tesztesetekből előállíthatjuk a tesztelési tervet
	\end{itemize}
\end{frame}


\begin{frame}
	\frametitle{Tesztelő eszköz megválasztása}
	\textbf{Szempontok}:
	\begin{itemize}
		\item ne kerüljön át a hangsúly a keretrendszer specialitásaira
		\vspace*{2px}
		\item legyen könnyen tanulható
		\vspace*{2px}
		\item egyszerű legyen az automatikus tesztkörnyezet kialakítása
		\vspace*{2px}
		\item jó dokumentáltság
		\vspace*{2px}
		\item C++
		\vspace*{2px}
		\item ne kelljen telepíteni, fejlesztői környezetbe integrálni; egy fejlécállományként legyen megvalósítva külső függőségek nélkül 
		
	\end{itemize}\pause
	\vspace*{15px}
	A választás a \textbf{Catch} eszközre esett.
\end{frame}

\begin{frame}
	\frametitle{Tapasztalatok}
	\begin{itemize}
		\item az intuíció alapján dokumentált tesztesetekkel szemben a szabványos szürke doboz tesztesetek hatásosabban szűrik ki a kódolási hibákat\vspace*{3px}
		\item az automatikusan futtatható tesztekkel a hallgatók által írt programok minősége javult\vspace*{3px}
		\item az automatikus tesztekkel ellátott kódrészek gyorsan ellenőrizhetők\vspace*{3px}
		\item a Catch használata gyorsan elsajátítható
	\end{itemize}
\end{frame}

\begin{frame}
	\frametitle{Szabványos tesztelés}
	\begin{center}
		\Large{Köszönöm a figyelmet!}
	
	\vspace*{25px}
	{\small A projekt az Európai Unió támogatásával, az Európai Szociális Alap társfinanszírozásával valósult meg (EFOP-3.6.3-VEKOP-16-2017-00002).}\end{center}
\end{frame}

\end{document}
