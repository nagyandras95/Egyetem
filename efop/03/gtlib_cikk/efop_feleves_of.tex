\documentclass[11pt]{beamer}

\usepackage[utf8]{inputenc}
\usepackage[magyar]{babel}
\usepackage[T1]{fontenc}
\usepackage{lmodern}
\usepackage{zi4}
\usepackage{multirow}
\usepackage{listings}
\usepackage{ragged2e} % For \justifying command

\usetheme{Warsaw}

\renewcommand\UrlFont{\ttfamily\footnotesize}

% no navigation symbols
\setbeamertemplate{navigation symbols}{} 

% frame numbers
\expandafter\def\expandafter\insertshorttitle\expandafter{%
  \insertshorttitle\hfill%
  \insertframenumber\,/\,\inserttotalframenumber}

\author{Nagy András}
\title{txtUML féléves eredmények}
\date{2019. január 31.}

\logo{%
	\makebox[0.95\paperwidth]{%
		\hfill%
		\includegraphics[width=3.2cm,keepaspectratio]{efop-logo.jpg}%
	}%
}

\begin{document}

\begin{frame}
\titlepage
\end{frame}


\begin{frame}
	\frametitle{Infodidact konferencia}
	\begin{itemize}
		\item Segítség a C++ osztálysablon könyvtár átdolgozássában.
		\item Tapasztalatok, újdonságok összegyűjtése a könyvtárral kapcsolatban.
		\item Eredmények publikálása.
		\item Előadás az Infodiadct konferencián Zamárdiban.
	\end{itemize}
\end{frame}

\begin{frame}
	\frametitle{Egyetem, szakma népszerűsítése}
	\begin{itemize}
		\item Codeweek hetén programozást, modellezést népszerűsítő előadás középiskolásoknak.
		\item Pályaorientáció hetén érvelés amelett, miért ne menjenek el érettségi után az iparba, tudományos élet népszerűsítése.
	\end{itemize}
\end{frame}

\begin{frame}
	\frametitle{Kihelyezési konfiguráció átdolgozása}
	\begin{itemize}
		\item Tegyük egyszerűbbé és hatékonyabbá a szálkezelést.
		\item Fix szálakkal dolgozzunk, amit vagy automatikusan a platform alapján határozunk meg, vagy a felhasználó mondja meg.
		\item Legyen konfigurálható, hogy egy szálkezelő egység az összes szál hanyad részét kapja meg.
		\item Ne hozunk létre futás közben új szálakat, és ne is szüntetünk meg, ami nagy overheaddel, sok szinkronizációval járt eddig. A terhelést eloszthatjuk az objektumok között.
		\item Protoítpus készítése, ami még finomításokra szorul.
	\end{itemize}
\end{frame}

\begin{frame}
	\frametitle{txtUML féléves eredmények}
	\begin{center}
		\Large{Köszönöm az efop támogatását!}
	
	\vspace*{25px}
	{\small A projekt az Európai Unió támogatásával, az Európai Szociális Alap társfinanszírozásával valósult meg (EFOP-3.6.3-VEKOP-16-2017-00002).}\end{center}
\end{frame}

\end{document}
