\documentclass[12pt,a4paper]{article}
\usepackage[utf8]{inputenc}
\usepackage[magyar]{babel}
\usepackage[T1]{fontenc}
\usepackage{amsmath}
\usepackage{amsfonts}
\usepackage{amssymb}
\begin{document}


\begin{itemize}
\item Egy szekvenciális inputfájlban a felszín egy vonalán adott távolságokként mért
tengerszint feletti magasságértékeket tárolunk. Milyen magas a legmagasabban fekvő
horpadás?
\item Gyűjtsük ki egy szekvenciális input fájlban rendezve tárolt egész számok közül, hogy melyik számból hány darab található a fájlban.
\item Egy szekvenciális inputfájlban a banknál számlát nyitott ügyfelek e havi kivét/betét
forgalmát (tranzakcióit) tároljuk. Minden tranzakciónál nyilvántartjuk az ügyfél
számlaszámát, a tranzakció dátumát és az összegét, ami egy előjeles egész szám
(negatív a kivét, pozitív a betét). A tranzakciók a szekvenciális fájlban számlaszám
szerint rendezetten helyezkednek el. Gyűjtsük ki azon számlaszámokat és az ahhoz
tartozó tranzakciónak egyenlegét, ahol ez az egyenleg kisebb –100000 Ft-nál!
\item Számoljuk meg egy karakterekből álló szekvenciális inputfájlban a szavakat úgy, hogy
a 12 betűnél hosszabb szavakat duplán vegyük figyelembe! (Egy szót szóközök vagy a
fájl vége határol.)
\item Másoljuk át karakterenként egy szekvenciális inputfájl szövegét egy outputfájlba úgy,
hogy a szavak között csak egyetlen szóközt tartunk meg! (HF)
\item Egy szekvenciális inputfájlban vadászok adott napi vadászatain elejtett zsákmányait
(fajtanév és súly párok formájában) vadászok szerint rendezetten tároljuk.
\begin{itemize}
\item Igaz-e, hogy minden vadász valamelyik vadászatán lőtt medvét?
\item Melyik vadász lőtte a legnagyobb medvét?
\item Hány olyan vadász volt, aki minden vadászatán lőtt nyulat és a legnagyobb szavas
zsákmánya meghaladta a 250 kilogrammot.
\end{itemize}
\end{itemize}

\end{document}