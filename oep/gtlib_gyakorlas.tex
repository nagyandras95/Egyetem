\documentclass[12pt,a4paper]{article}
\usepackage[utf8]{inputenc}
\usepackage[magyar]{babel}
\usepackage[T1]{fontenc}
\usepackage{amsmath}
\usepackage{amsfonts}
\usepackage{amssymb}
\begin{document}

A feladatokat gtlibbel kell megoldani, nem lehet elágazásokat és ciklusokat használni, ezeket egy meglévő könyvtárkomponensre kell visszavezetni. Használd GT honlapját segítségnek, mivel a ZH-n csak ez lesz elérhető. \\

Konfiguráció \textit{CodebBocks}-ban: 
\begin{itemize}
\item Töltsd le libraryt gt honlapjáról.
\item Hozz létre egy új projektet.
\item Másold be a projekt fájl mellé a könyvtárban található fejállományokat.
\item Add hozzá őket a projekthez (Project, Add Files..). 
\item Kapcsold be az std=c++11-et (Project, Build Options..)
\end{itemize}

Feladatok:
\begin{enumerate}
\item Bemelegítő feladat, FSZ képzésen hármasért: A fájl egy sora egy hallgató nevét, és félév végén szerzett plusz-minuszok össz eredményét tartalmazza (egész szám). Igaz-e, hogy senki nem bukott meg (minden hallgatónak legalább nulla az összeredménye)? \\
Pl: \\
h1 4 \\
h2 0 \\
Válasz: Igaz
\item BSc-n hármasért, FSZ-en ötösért: A fájl egy sora egy hallgató nevét, utána a félév folyamán írt plusz-minuszok eredményeit tartalmazza (-1, 0, 1). Igaz-e, hogy senki nem bukott meg (minden hallgatónak legalább nulla az összeredménye)? 
 \\
Pl: \\
h1 -1 0 -1 1 \\
h2 1 1 -1 0  \\
Válasz: Hamis
\item BSC-n ötösért: A fájlok sorainak formátuma nem változik az előzőhöz képest, viszont egy-egy hallgatónak több sorai is lehet, minden sor egy adott tárgyból elért plusz-minuszokat sorolja fel. Igaz-e, hogy  minden hallgatónak a legnagyobb össz eredménye nagyobb-e, mint 0? 
 \\
Pl: \\
h1 -1 0 -1 1 \\
h1 0 0 -1 \\
h1 0 1 1 \\
h2 1 1 0 \\
h2 1 -1 -1 0  \\
Válasz: Igaz
\end{enumerate}


\end{document}