\documentclass[12pt,a4paper]{article}
\usepackage[utf8]{inputenc}
\usepackage[magyar]{babel}
\usepackage[T1]{fontenc}
\usepackage{amsmath}
\usepackage{amsfonts}
\usepackage{amssymb}
\begin{document}
Állapotgépes feladatok

\begin{enumerate}
\item Egyszerűsített billentyűzet modellezése: ha a CapsLock aktivált, akkor minden más billentyű lenyomásra nagybetűs karaktereket, különben kisbetűs karaktereket kapunk. (Legyen két állapot: CapsLock aktív illetve inaktív. Legyen háromféle művelet: CapsLock lenyomása, más gomb lenyomása, kikapcsolás.)
\item Video lejátszó: Készítsük el egy magnó osztálydiagramját és állapotdiagramját a következő
leírás alap ján! A magnóban található egy fej és egy motor, amelyeket négy
gomb segítségével vezérelhetünk. A gombokat elegendő megérinteni a vezérlés
során. \\ 
A négy gomb és vezérlési szerepük: \\
Állj: leállítja a motort, és a fejet leveszi a szalagról, ha az azon volt \\
Lejátszás: Lejátszó sebességbe helyezi a motort, és a fejet a szalagra
helyezi \\
Előre: A motor előre tekeri a szalagot. \\
Hátra: A motor hátra tekeri a szalagot. \\

\item Egy bankautomata a következő módon működik. Azzal indul, hogy az ügyfél behelyezi a kártyáját, majd beviszi a pinkódot, amivel háromszor próbálkozhat (harmadik sikertelen kísérlet után a tranzakció elutasítva). Ha sikeres a pinkód megadása, akkor le lehet kérdezni az egyenleget, vagy ki lehet venni pénzt az automatából. Ha a megadott összeg nagyobb, mint az egyenleg, akkor sikeres a pénzkivét, különben nem.

\item Egy közlekedési lámpán piros, piros-sárga, zöld, sárga fények vannak. A lámpa 60 másodpercig piros és 90 másodpercig zöld színű. Az átmeneti állapotok 5 másodpercig tartanak: pirosról a zöldre a piros-sárgán keresztül, zöldről a pirosra a sárgán keresztül. Kezdetben a lámpa piros.


\item Verem állapotgépe 
Vegyünk egy korlátos vermet, amelyet egy (t:array[0..max-1] of Elem) tömb, és egy (top:int) index reprezentál.  Három állapotot vezetünk be: „üres” (top=-1), „normál” (-1<top<max-1), „tele” (top=max-1), amelyek között a verem műveletek (push(), pop()) hatására következik be átmenet. Kezdetben (kezdeti átmenet) top:=0. 

\end{enumerate}

\end{document}
