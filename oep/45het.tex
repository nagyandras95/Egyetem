\documentclass[12pt,a4paper]{article}
\usepackage[utf8]{inputenc}
\usepackage[magyar]{babel}
\usepackage[T1]{fontenc}
\usepackage{amsmath}
\usepackage{amsfonts}
\usepackage{amssymb}
\title{Feladatok nevezetes felsorolókkal}
\begin{document}
\maketitle

\begin{itemize}
\item Igaz-e, hogy minden szám páros egy egész számok szekvenciális inputfájljában? 
\item Egy szekvenciális fájlban egy bank számlatulajdonosait tartjuk nyilván (azonosító, összeg) párok formájában. Adjuk meg annak az azonosítóját, akinek nincs tartozása, de a legkisebb a számlaegyenlege!
\item Egy  szekvenciális  inputfájlban  egyes  kaktuszfajtákról ismerünk  néhány  adatot:  név, őshaza,  virágszín,  méret.  Válogassuk  ki  egy  szekvenciális  outputfájlba  a  mexikói,  egy másikba a piros virágú kaktuszokat!
\item Összegezzük egy mátrixnak a sakktábla fehér mezőnek megfelelő elemeit!
\item Egy szekvenciális inputfájlban tárolt napi hőmérsékleteknek 
\begin{itemize}
\item mennyi az átlaga az első fagypont alatti értéket megelőző részben?
\item mennyi az első fagypont alatti napot követő napok átlaga az első fagypont alatti napot is figyelembe véve?
\item mennyi az átlaga az első fagypont alatti értéket megelőző és az azt követő napokra vetítve külön-külön?
\end{itemize}
\item Egy szekvenciális inputfájlban a felszín egy vonalán adott távolságokként mért tengerszint feletti magasságértékeket tárolunk. Milyen magas a legmagasabban fekvő horpadás?
\item Adott egy számítástechnikai üzlet aznapi forgalma: számlák sorozata, ahol egy számlán szerepel a vásárló neve, az általa vásárolt termékek neve és ára. Mennyi az aznapi bevétel?
\end{itemize}

\end{document}