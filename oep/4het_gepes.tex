\documentclass[12pt,a4paper]{article}
\usepackage[utf8]{inputenc}
\usepackage[magyar]{babel}
\usepackage[T1]{fontenc}
\usepackage{amsmath}
\usepackage{amsfonts}
\usepackage{amssymb}
\title{Negyedik gépes óra feladat}
\begin{document}

\maketitle

Adott egy számítástechnikai üzlet aznapi forgalma: számlák sorozata,ahol egy számlán szerepel a vásárló neve, az általa vásárolt termékek neve és mennyisége. Összesen hány terméket adtak el aznap? \\

Megoldás gondoldatmenete:
\begin{itemize}
\item Készítünk egy felsorolót, amit egy fájlnév megadásával tudunk konstruálni, és felsorolja, hogy egy-egy sorban összesen hány terméket adtak el.
\item Ezután már egy egyszerű végig iterálással és összegzéssel megoldható a feladat.
\item Egy-egy sor beolvasása C++-ban bonyolultabb feladat, mivel változó hosszú sorokat kell feldolgozni. Ehhez használjuk a \textit{getline} függvényt, mely egy \textit{string}-be helyezi a fájlnak egy sorát, majd a \textit{stringstream} osztályt, mellyel végig tudjuk olvasni a \textit{string}-et a $>>$ operátorral. 
\end{itemize}

\end{document}