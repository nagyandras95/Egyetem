\documentclass[12pt,a4paper]{article}
\usepackage[utf8]{inputenc}
\usepackage[magyar]{babel}
\usepackage[T1]{fontenc}
\usepackage{amsmath}
\usepackage{amsfonts}
\usepackage{amssymb}
\title{Objektumelvű programozás - 1. heti feladatok}

\begin{document}

\maketitle

\begin{itemize}
\item Két nem-negatív szám szorzatának kiszámolása összeadásokkal.
\item Természetes szám faktoriálisának kiszámolása.
\item Hatvány kiszámolása összeszorzásokkal. (HF)
\item Tömbben tárolt számok abszolút értékeinek összege. (csak említsük meg és HF)
\item Két vektor skaláris szorzata. (HF)
\item Számoljuk össze egy tömb páros elemeit! (A számlálás egy speciális összegzés)
\item Van-e egy n elemű tömbben páros szám? (Lineáris kereséssel helyett számlálással).
\item Egy n elemű tömb minden eleme páros-e? (linker helyett összeéselés)
\item Válogassuk ki a páros elemeket egy sorozatba illetve egy halmazba!
\item Természetes számokat tartalmazó tömb maximális elemének meghatározása (még ez
is megy összegzéssel, de a minimális elem keresése már nem, hiszen annak nincs
baloldali nulla eleme a természetes számokon).
\end{itemize}


\end{document}