\documentclass[12pt,a4paper]{article}
\usepackage[utf8]{inputenc}
\usepackage{amsmath}
\usepackage{amsfonts}
\usepackage{amssymb}
\usepackage{graphicx}
\author{Nagy András}
\title{Webes alkalmazások fejlesztése - 1. beadandó}

\begin{document}
\maketitle

\section{Feladat}
Készítsünk egy aukciókkal foglakozó online rendszert, ahol különböző tárgyakra
licitálhatnak a felhasználók.
A  webes  felületet  a  licitálók  használhatják  a 
tárgyak  megtekintésére,  illetve 
ajánlattételre.
\begin{itemize}
\item A  főoldalon  a  legutoljára  meghirdetett  20 tárgy listázódik (név,  hirdető,jelenlegi licitösszeg), de lehetőségünk van kategóriánként megtekinteni az összes(még  aktív)  hirdetést. Egy  oldalon  legfeljebb  20  tárgy  látható  (a 
meghirdetés dátuma szerint csökkenő sorrendben), 
az oldalak között lapozni lehet. A lista szűrhető név(
részlet)-re. A tárgyat kiválasztva megjelennek a 
részletes adatok (kép, leírás, lezárás és meghirdetés 
dátuma , aktuális licit).
\item A licitálónak előbb regisztrálnia kell az oldalon (név, telefonszám,e-mail cím,felhasználónév, jelszó, megerősített jelszó), majd ezt követően bejelentkezhet. A bejelentkezett felhasználó kijelentkezhet.
\item Bejelentkezést  követően érhető  el  a  licitálás  minden  aktív  tárgynál. Licitáláshoz ki 
kell jelölni a tárgyat és meg kell adni az összeget. Első licit esetén az  összegnek a  minimális  licitnek  kell  lennie, később  pedig mindenképpen nagyobbnak kell lennie a korábbi liciteknél. Egy felhasználó tetszőlegesen sokszor licitálhat egy tárgyra. A licitet visszavonni nem lehet.
\item A felhasználó külön listázhatja azokat a tárgyakat, amelyekre legalább egyszer 
licitált. A listában külön megjelöljük az aktív tárgyakat, valamint azokat, ahol 
vezeti a licitet
\end{itemize}

\section{Feladat elemzése}
A feladatot a három rétegű, Model-View-Controller architektúrában valósítjuk meg.
\begin{itemize}
\item Az adatok tárolásához, és a felhasználó kezeléshez létrehozunk egy adatbázist, melyben a tárgyak, a licitek, valamit a felhasználók adatait tárolhatjuk. Az adatbázis modelljét az AuctionPortalEntities entitás modell biztosítja.
\item Az oldal egységes megjelenítésért létrehozunk egy megosztott megjelenítőt, melyben be és kijelentkezhetünk, láthatjuk a tárgyakat kötegórák szerint és szűrhetünk is rájuk.
\item A Home könyvtár alatt létrehozzuk a tárgyak listázásért felelős nézetet, valamint az adott tárgy részletei nézetet.
\item Külön nézetet hozunk létre a licitáláshoz, a regisztrációhoz és a bejelentkezéshez. Mindezekhez létrehozunk egy-egy nézetmodellt, melyben eltárolhatjuk a felhasználó által kitöltött információkat.
\item A kontrollerek felelőssége csökkentése érdekében létrehozunk két funkcionalitásért felelős osztályt, az AccounstServicet-t, mely a felhasználói műveltekért felelős, valamint az AuctionService-t, amely a konzisztens licitálást biztosítja, valamint a megfelelő licitek és tárgyak elérést biztosítja.
\item Azon funkciókat, melynek minden kontrollernek  végre kell hajtania, mint a ketegóriák meghatározása, kiemeljük egy bázis kontrollerbe.
\item Külön kontroller felel a felhasználói műveletekért, valamint a licitálás kezeléséért.
\end{itemize}

\section{Felhasználói estek}
\includegraphics[scale=0.5]{felhasznaloi_diag.jpg}

\section{A rendszer szerkezete}
\subsection{Komponensek}
Az MVC acthitektúrának megfelelően a programnak 3 fő komponense van.
\begin{itemize}
\item A Models névtér tartalmazza az adatbázis alapján generált entitásmodellt, a nézetmodelleket, az felhasználókezelését , valamint az adatbázis műveltekért felelős osztálymodelleket, valamint a validációért felelős statikus osztályokat.
\item A Contorllers névtéren belül a kontrollerek helyezkednek el. A program 3 különböző kontrollere oszlik, valamint a közös funkcionalitást biztosító bázis kontrollere.
\item A View-en belül helyezkedik el a közös megjelenítésért felelős nézet, a tárgyak listázásáért felelős főoldali nézet,  a be-ki jelenkezesértét és a regisztrációért felelős nézetek, valamint a tárgyak részleteit megadó, és a licitálás felület.
\end{itemize}
\subsection{Osztályok}
\includegraphics[scale=0.5]{osztaly_szerkezet.jpg}

\section{Adatbázis felépítése}

\includegraphics[scale=1.0]{entity_connect.jpg}

\end{document}